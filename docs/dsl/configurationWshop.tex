\subsection{Schema Definition}% via \dcl~and Datalog}
\label{configuration}
Recall that some initial patterns are mapped to the basis via configuration defined using \dcl~(\dclfull). Consider the earlier \dcl~example to define a basis pattern called \texttt{\#person}:\\ %(reproduced below) 
%Consider the example of Section~\ref{terminology}:\\\\
\phantom{.}~~~\dclcode{person(name:String, age:Int, city:String)~~~// \dcl~code}

In reality, the above will give an error because in addition to attributes, each pattern must have either one or more {\em primary keys} or one or more {\em pattern keys}. %These are essentially attributes with special meaning. 

\subsubsection{Primary Keys} 
\label{primarykeys}
%A primary key is equivalent to its counterpart in SQL - 

Two patterns are linked via primary keys. Any attribute can be set as a primary key by appending an ID enclosed in square brackets to the attribute, as in the following \dcl: \\ %. Therefore the following is valid \dcl~code (won't give an error):\\
{\dclcode{\phantom{~}~person(persID:String[pID], name:String, age:Int, city:String)}}

The above defines a primary key with ID \texttt{pID}, written as \texttt{[pID]}. This ID is used to match primary keys of other patterns. 
%(the identifier \texttt{persID} is never used).
Now let us define two more patterns via \dcl: \\
{ 
%\small
\dclcode{\phantom{~}~corp(corpID:String[cID], name:String, founder:String)}\\
\dclcode{\phantom{~}~works(persID:String[pID], firmID:String[cID], empID:String)}
}

Here \texttt{\#corp} has one primary key \texttt{[cID]}, while \texttt{\#works} has two primary 
 keys, \texttt{[cID]} and \texttt{[pID]}. The two patterns are connected via the common primary key \texttt{[cID]}, similar to the way RDBMS tables are connected. % via primary-foreign key relationships. 
However, there are no foreign keys in \dsl~and all such relationships are {\em many-to-many}. 
Primary keys have the following properties:
\begin{enumerate}
	\item Primary keys attributes cannot be used in a filter. They are used internally for traversal and merging. %ing patterns.
	%\item patterns connected via primary keys have a {\em many-to-many} relationship. 
	\item A pattern returns its primary keys (or pattern keys - see below). Therefore, \texttt{\#person} will return \texttt{[pID]}s. 
	\item A new pattern inherits the (primary or pattern) keys of its parent pattern(s). Therefore, any pattern defined using \texttt{\#person} will also return \texttt{[pID]}s. 
\end{enumerate}
%\textbf{Example.} 
%
%Now let us define two more patterns via \dcl: \\
%{ 
%%\small
%\dclcode{corp(corpID:String[cID], name:String, founder:String)}\\
%\dclcode{works(persID:String[pID], firmID:String[cID], empID:String)}\\
%}

\textbf{Traversal:} Traversal occurs when a pattern is extended and the filter accesses an attribute of a different pattern. The patterns must be connected by a chain of primary keys for traversal to be successful (otherwise an error will be thrown). If such a path exists then \dsl~will automatically join the relations (i.e., pattens) in the path using the primary keys.
As an example, the following pattern, \texttt{\#fromAcme} uses \texttt{\#corp} directly and \texttt{\#works} indirectly for traversal:\\ \colorbox{Light}{\small \texttt{def \#fromAcme as \bfseries{\#person where \{\#corp.@name='Acme'\}}}}.

 Traversal from \texttt{\#person} to \texttt{\#corp} is done via \texttt{\#works} as follows:
	{\small \texttt{\#person[pID] -> \#works[pID,cID] -> \#corp[cID]}}. 	
	The traversal is explicitly seen in the compiled Datalog code:\\	
	{\small \texttt{fromAcme(pID) :- person(pID, \_, \_, \_), works(pID, cID, \_),\\\phantom{1}~~~~~~~~~~~~~~~~corp(cID, 'Acme', \_).}}

	 %\texttt{\#fromAcme}'s keys are derived from its parent pattern (\texttt{\#person}). Thus, \texttt{\#fromAcme} has the primary key \texttt{[pID]}. %This attribute is hidden from the end-user because it is a primary key. 
	
	
	\subsubsection{Pattern keys} 
	\label{patternkeys}
	
	Pattern keys define attributes that map to another pattern (see Example~\ref{patterkeyexample}, Section~\ref{description}).
		%We cannot use a primary key for this, since the corresponding attribute is invisible. 
		Pattern keys are defined by % by appending 
	an ID enclosed in  braces. % to the attribute. 
	The following %\dcl~code 
	defines a pattern with two pattern keys both with ID \texttt{pID}:\\%, which we will write as \texttt{\{pID\}}:\\\\
	 \dclcode{\phantom{1}~~parent(person:String\{pID\}, child:String\{pID\})}
	
	A pattern key has the following features: %is like a primary key with some differences:
\begin{enumerate}
	\item A pattern key attribute can be used in a filter. It can map to any pattern that returns a primary key with the same ID. In our example, both \texttt{@child} and \texttt{@person} attributes of \texttt{\#parent} can map to any pattern that returns \texttt{[pID]} (such as \texttt{\#fromAcme}). Therefore, we can, for example, write: \\
	{\small \colorbox{Light}{\texttt{def \#AcmeParent as \#parent where \{@child = \#fromAcme\}}}}
	%	because \texttt{\#fromAcme} outputs \texttt{[pID]}.
	\item In a filter, a pattern returning a pattern key can be used in place of a pattern returning the same primary key. Therefore, in the code above the \texttt{@child} attribute can also take as input any pattern returning \texttt{\{pID\}}.
	\item Pattern keys are non-traversable and cannot be merged with primary keys. Thus, a pattern returning \texttt{[pID]} cannot be merged with one returning \texttt{\{pID\}}. %The next version of \dsl~will allow such joins provided the patterns return a single key.

\end{enumerate}

\textbf{Hiding returned pattern keys.} A pattern returns all its pattern keys. For instance, \texttt{\#parent} as defined above will return the pattern keys \texttt{\{pID, pID\}}. %Therefore, the \texttt{\#dependent} pattern defined above also returns same pattern keys because it uses \texttt{\#parent} in its \texttt{<match>}. 
Suppose, we want it to return only the first key (representing parents), we can skip the other key(s) by appending `\texttt{!}' to it, as in:\\
\dclcode{\phantom{1}~~~parent(person:String\{pID\}, child:String\{pID!\})}
%\phantom{1}~~~~~
%making \texttt{\#parent} return only the first \texttt{\{pID\}}. 

After this, \texttt{\#parent} will return only the first key \texttt{\{pID\}}, and  the following code, for example, will be valid \dsl:\\ %\\\\
 \colorbox{Light}{\small \texttt{def \#grandParent as \bfseries{\#parent where \{@child = \#parent\}}}}

%\subsubsection{Reduced basis}
%\label{reducedbasis} 
%Let us enumerate the basis patterns for the examples: \\
%{\small
%	\dclcode{\phantom{1}~~~person(persID:String[pID], name:String, age:Int, city:String)}\\
%	\dclcode{\phantom{1}~~~works(persID:String[pID], firmID:String[cID], empID:String)}\\
%	\dclcode{\phantom{1}~~~corp(corpID:String[cID], name:String, founder:String)}\\
%	\dclcode{\phantom{1}~~~parent(person:String\{pID\}, child:String\{pID\})}
%	}
%	
%The data for these patterns is supplied by the user. All such patterns form the  {\em reduced basis}.
%
%\subsubsection{Extended Basis}
%\label{extendedbasis} 
%Suppose we need to define patterns for siblings and ancestors. %, which cannot be done from \dsl. Therefore, 
%Let us add them to the basis via \dcl:\\
%{\small
%\dclcode{\phantom{1}~~~sibling(left:String\{pID\}, right:String\{pID\})}\\
%\dclcode{\phantom{1}~~~ancestor(ancestor:String\{pID\}, descendant:String\{pID\})}
%}
%
%Their data can be inferred from the reduced basis and should not be user-supplied. These form the {\em extended basis}.
%
%\subsubsection{Initial Rules}
%\label{datalogrules}
%
%Since the extended basis does not contain data, using it in \dsl~will return no results. 
%Therefore, some rules need to be supplied to populate it using the reduced basis. These are the {\em initial rules}. 
%
%In the case of Datalog, our initial rules would be:\\
%{\small
%\texttt{\phantom{1}~~~sibling(x,y) :- parent(p,x),~parent(p,y),~x != y.}\\
%\texttt{\phantom{1}~~~ancestor(x,y) :- parent(x,y).}\\
%\texttt{\phantom{1}~~~ancestor(x,y) :- ancestor(x,a), ancestor(a,y).}
%}
%
%In the case of SQL, our initial rules would map to views:\\
%{\small
%\texttt{\phantom{1}~~~sibling(x,y) :- parent(p,x),~parent(p,y),~x != y.}\\
%\texttt{\phantom{1}~~~ancestor(x,y) :- parent(x,y).}\\
%\texttt{\phantom{1}~~~ancestor(x,y) :- ancestor(x,a), ancestor(a,y).}
%}
%
%Once the reduced basis, the extended basis and the initial rules are defined, the configuration is complete. The configuration is validated as described in Appendix~\ref{validation}.
%
%A basis pattern can be hidden using `!' as in:\\
%\dclcode{works!(persID:String[pID],corpID:String[cID], empID:String)}
%
%Hidden basis patterns are used only for traversal.
%After this \texttt{\#works} will cannot be used in queries but may be used internally for traversal.

%\phantom{.}~~~\dclcode{person ! (name:String, age:Int, city:String)~~~// \dcl~code}


% and \dsl~queries can be issued. 
%Some additional configuration settings are given in Appendix~\ref{config}. 
Once the pattern schema, the primary and pattern keys are defined in \dcl, the schema definition is complete. The set of initial patterns defined in the schema is called the {\em basis}. Advanced users can make additional tweaks to the basis configuration as described in Appendix~\ref{basis}.

A \dsl~server will be initialized with a basis along with raw data for its patterns (supplied in CSV format). 
After this, it can accept \dsl~queries and return data based on the result mapping supplied. The \dsl~and the basis must adhere to semantic restrictions given in Section~\ref{semantic}. 

Since \dsl~is primarily a query language, it does not have queries to insert data and all external data must be provided in the initialize phase. However, every \texttt{def} query in \dsl~effictively creates a temporary pattern internally. 


%Section~\ref{compilerlogic} describes the compiler. 