%%\documentclass{letter}
%\documentclass[conference]{IEEEtran}
%\documentclass{sigplanconf}
\documentclass{acm_proc_article-sp}
%\documentclass{llncs}
%\documentclass{sig-alternate}
\usepackage{amsmath}

\usepackage{rotating}
\usepackage[obeyspaces]{url}
\usepackage{multirow}
\usepackage{amssymb}
\usepackage[T1]{fontenc}
\usepackage[lighttt]{lmodern}
\usepackage{color}
\definecolor{Light}{gray}{.90}
\definecolor{Faint}{gray}{.96}
\usepackage[colorlinks=true,linkcolor=blue,citecolor=blue]{hyperref} % uncomment for pdf
\newcommand{\dsl}{DQL}
\newcommand{\dslfull}{\underline{D}ata \underline{Q}uery \underline{L}anguage}
\newcommand{\dcl}{PDL}
\newcommand{\dclfull}{\underline{P}attern \underline{D}efinition \underline{L}anguage}
\newcommand{\tld}{\raise.17ex\hbox{$\scriptstyle\sim$}}
\newcommand{\dclcode}[1]{\textsl{\textsf{#1}}}
\newcommand{\ignore}[1]{} % may contain useful stuff (that needs more work)

\newif\iffull
\fulltrue % or
%\fullfalse
\newcommand{\full}[1]{
\iffull
  #1
\else
  % electronic
\fi
}

\begin{document}
%\title{Finding Bugs and Coding Anti-Patterns Using Datalog %\dsl 
\title{Data Query Language: A DSL for Vetted Queries} 
\author{DQL Developers}
\maketitle
\begin{abstract}

One of the main goals of OPAL (Open Algorithms) project is to enable seamless data sharing between arbitrary entities, while at the same time ensuring that query results do not leak sensitive information. For instance, if an identity provider queries a data store, it should be able to obtain results that allow it to create assertions about an identity (such as {\em id A belongs to group G}) and nothing else. In particular, it should not be able to obtain personally identifying information about the individual.
OPAL attempts to do this by restricting the type of allowed database queries. Additionally, OPAL aims to bring together heterogeneous data stores, each with its own query languages. There are several dialects even with SQL. Due to this diversity, defining a common framework for privacy preserving data sharing seems difficult because a query auditor will have to understand several syntax and semantics. To address this, OPAL defines its own query language called DQL (Data Query Language) which is data-store agnostic. 
Similarly to SQL, DQL can be used to represent arbitrary relationships between data, create more views of the data and finally query those views. DQL is expressive, yet compact and can capture SQL queries spanning several lines in a single statement. Another goal of DQL is to allow us to prove certain assertions about information leakage. 

\end{abstract}

\input{dsl/introWshop.tex}
\input{dsl/backgroundWshop.tex}
\section{Introduction}

The OPAL project aims to enable competing organizations to come together and share data in order to achieve some common goal. The health sector (for instance cancer research) seems to me one benefactor of such technology. Data sharing in clinical trials for cancer research is governed by strict privacy laws and consequently a large amount of time and money is spent on repetitions of trials generating similar data. If such data was already available, many of these tests could be skipped. Thus, any data-sharing in such scenarios has to go through extensive scrutiny for possible leakage. Not only does this entail technical skills but also legal expertise. OPAL aims to address this issue by requiring:
%OPAL addresses this issue by requring that 
\begin{enumerate}
	\item Queries should return only aggregate answers and never information about individual rows.
	\item Only vetted queries should be allowed. A vetter would publish templates of allowed queries.
	\item A query auditor analyzes queries created from the vetted templates to ensure no private information is leaked.
	\item The entire transaction of query/response is be recorded in a blockchain for audit trails.
\end{enumerate}

%A query auditor is a program that takes in a query along with a sequence of past query-responses and decides if the response to the current query can leak any sensitive data.
\subsection{Query Auditing}
The following definitions are taken from the literature. A {\em dataset} is a set $X$ of labeled real numbers $\{x_i, x_2,\ldots x_n\}$ where $n$ is public. 
A query $q$ consists of a pair $(f, s)$, where $f\in \{\textsf{max}, \textsf{min}, \textsf{avg}, \textsf{sum}\}$ representing the aggregate functions and $s\in[1..n]$. 
The response to $q$ is a real number representing the aggregate applied on the subset of $X$ defined by $s$. An {\em offline auditor} takes in a sequence of query-responses and decides if the responses leak any sensitive data. An {\em online auditor} takes in a sequence of query-responses and additionally a query $q$ along with its response $r$ to decide if $r$ might leak sensitive data (it does not care if the past responses leaked information). If an online auditor decides that $r$ leaks information then $r$ can be denied. However, as shown in~\cite{auditor}, the very act of denying a response can leak information. A {\em simulable auditor} is an online auditor that takes this decision without looking at $r$. Hence, an adversary can also simulate the auditor's role and we can be ensured that denials do not leak information. 

The concept of information leakage is either captured via {\em full disclosure}, where an adversary is able to compute some or all of $x_i$s or {\em partial disclosure}, where the adversary is able to narrow down on the range of some (or all) $x_i$s. Note that the full disclosure definition is too weak to be useful in practice because even if the adversary cannot compute the exact value, a narrow enough range may be sufficient for leakage. On the other hand, partial disclosure is too strong to be achievable in practice, because every useful query must leak {\em some} range information. Note that we do not consider the trivial case of denying all queries since that indeed achieves resistance against partial disclosures.

\subsection{Query Vetting in OPAL}

OPAL uses slightly modified definitions. An {OPAL dataset} is a set $X$ of labeled real numbers $\{x_i, x_2,\ldots x_n\}$ where $n$ is \textbf{secret}. An OPAL query $q$ consists of a pair $(g, t)$, where $g\in \{\textsf{max}, \textsf{min}, \textsf{avg}, \textsf{sum}, \textsf{count}\}$ representing the aggregate functions and $t\in A$ for some $A\subsetneq 2^{[1..n]}$, the power set of $[1..n]$. We call $A$ the {\em set of allowed queries} of $X$. 

An {\em OPAL query template} is a mapping from the data set $X$ to an allowed query set $A$. An {\em OPAL vetter} is responsible for defining the query template, while an {\em OPAL auditor} decides if a given query $q$ is indeed an element of $A$.
%vetted queries, query aditors

The OPAL definition are more usable because typical queries such as \texttt{select max(age) from users where state = 'MA'} do not allow the adversary to select arbitrary subsets of $X$ but rather those defined by a filter (\texttt{state = 'MA'}). Thus, one approach to query vetting would be to restrict the filters that can be applied. For instance, a filter to ensure that $n$ remains secret would need to ensure that the adversary can never select all or exclude a fixed number of elements from a query set.

\subsection{OPAL Query Language}

In order to define such a query framework and filter rules for allowed queries, we use a query language called DQL, which is specifically designed for working with sets and filters. The prototype implementation of DQL uses Datalog as its underlying query engine but adapters for converting to standard SQL can be created. The key point is that once a DQL query is considered safe by a vetter or auditor, we don't have to worry about vetting the corresponding SQL or another low-level query. Hence we can focus our security analysis on DQL while perform the actual query in another language if needed. 

\section{Overview of DQL}

\dsl~is a language for querying structured data stored in a set of {\em tables}. A table in \dsl~is called a {\em pattern}. 
A pattern has a name, a {\em schema} describing the data and zero or more rows containing the data. %Thus, a pattern directly maps to a table in SQL or a relation in Datalog. 
Pattern names start with \texttt{\#}, such as \texttt{\#users}. 
\dsl~queries are of two types: a \texttt{def} query defines new patterns (similar to creating views in SQL) and a \texttt{find} query returns row IDs of a pattern (similar to a select query in SQL). 

%Using DQL, a programmer can define new tables from the initial ones (using a \texttt{def} query) and finally obtain the results of a desired table (using a \texttt{find} query).
%Under the hood, \dsl~uses Datalog and it is useful (but not necessary) for the reader to have some knowledge of it. Appendix~\ref{datalog-intro} gives a brief overview of Datalog.
\textbf{Schema:} A data store makes its data available by first publishing the schema in \dcl~(\dclfull), an example of which is: 

%For instance, a SQL database will expose certain tables (or a view of them) via initial patterns defined using PDL.
%DQL queries can combine PDL code from several data stores and present a combined view to the querier (who may not even be aware that each initial pattern maps to a table in a different database).

{\small
	\dclcode{users(name:String, userID:String[uID], stateID:String[sID], age:Int)}\\
	\dclcode{state(name:String, id:String[sID], capital:String)}
}

The above defines two patterns \texttt{\#users} and \texttt{\#state} with various attributes. Additionally, the \texttt{stateID} attribute of \texttt{\#users} and the \texttt{id} attribute
of \texttt{\#state} can be used in a join because they map to the common key \texttt{[sID]}. This published schema is called the {\em basis}. 

\textbf{Comparing with SQL:} Since \dsl~is strictly a query language, it does not provide methods to insert or update rows. However, note that a \dsl~\texttt{def} query is 
equivalent to creating a view in SQL. Similar to SQL, \dsl~is a {\em set algebra}, a language that operates on sets and provides the ability to perform operations on arbitrary subsets of those sets. In SQL however, the programmers needs to specify {\em how} to do the operations by specifying the joins. In \dsl, the joins are automatically inferred, allowing the programmer to focus on {\em what} rather than {\em how}. As an example, the \dsl:\\
\texttt{\phantom{~}find \#users where \{\#state.@capital = `Springfield`\}} 

can map to the SQL:\\
\texttt{\phantom{~}select * from users where }\\
\texttt{\phantom{~}~~~~~~~~~~user.stateID = state.stateID and}\\
\texttt{\phantom{~}~~~~~~~~~~state.capital = 'Springfield'}.

The joins are automatically inferred from the DQL schema.
%The PDL \dclcode{Users(name:String, agE:Int)} defines a SQL table Users(name VARCHAR(255), age INT)
%Compared to SQL, we have constrained DQL 
%can be mapped to any SQL or Datalog nderlying database. 

DQL is expressive enough to represent all necessary relationships found in most SQL databases and additionally allows recursive rules to be defined 

%There are two ways to use \dsl~with OPAL. One is to write an adapter for \dsl~and each database. A second and more pragmatic solution would be to use a database engine that directly accepts \dsl~queries. As a proof of concept for the second approach, we have implemented a database engine called Open\dsl~that accepts structured data in CSV format and uses \dsl~as its query language. 

%Under the hood, Open\dsl~uses Datalog and it is useful (but not necessary) for the reader to have some knowledge of it. Appendix~\ref{datalog-intro} gives a brief overview of Datalog.
 
 %Open\dsl~internally uses Datalog for its query engine.
%

%\textbf{Tables:} In DQL, a homogeneous representation of data is called a {\em table}. %Thus, a pattern directly maps to a table in SQL or a relation in Datalog. 
%A DQL query generates new tables (similarly to creating views in SQL) or returns row IDs of a table (select query in SQL). 

%\textbf{Schema:} A data store makes its data available by first publishing its schema in \dcl~(\dclfull), an example of which is given above in italics. 
%This published schema, the structure of a set of initial tables in \dcl~, is called the {\em basis}. 
%%For instance, a SQL database will expose certain tables (or a view of them) via initial patterns defined using PDL.
%%DQL queries can combine PDL code from several data stores and present a combined view to the querier (who may not even be aware that each initial pattern maps to a table in a different database).

%Using DQL, a programmer can define new tables from the initial ones (using a \texttt{def} query) and finally obtain the results of a desired table (using a \texttt{find} query).

\section{The \dsl~Framework}
\label{overview}


%\subsection{Overview}
%\textbf{Features.} The following are some key features of \dsl: %~that enable us to attain the above design goals.
%\begin{enumerate}
	%%\item {\em No access to Datalog relations:} \dsl~does not allow direct access to the underlying Datalog relations. Rather it encapsulates the relations into a \emph{pattern}. 
	%%A pattern needs to be defined %(either via \dsl~code or via configuration) 
	%%before it can be used. 
	%\item {\em No predefined patterns:} \dsl~does not have any predefined patterns. Rather, it allows users to define their own initial patterns, called {\em the basis}.
	%%For example, in the context of static analysis, 
	%%a basic pattern could match all instructions inside a loop, while a complex pattern would match only JDBC calls in a loop (see Example~\ref{sql:loop} in Section~\ref{jdql:examples}). 
	%\item {\em Extensible:} An advanced user has ability to change the basis
	%if it is found that existing patterns are not able to capture the domain properly. 
	%%\item {\em Validation:} \dsl~is type-safe and validates any hand-written rules 
	%%to eliminate potential bugs (Section~\ref{validation}). 
	%%\item {\em Type-safety:} \dsl~enforces type correctness. % and catches invalid types. 
	%%So, for example, if a variable is of type \texttt{Int}, assigning any other type to it will cause an error. 
	%%\item {\em RDBMS Emulation:} In \dsl, patterns are similar to RDBMS tables that are linked via {\em primary keys}. 
	%%There are no foreign keys; a somewhat related notion of {\em pattern keys} is used. Furthermore, \dsl~syntax is closer to SQL  and  more friendlier than Datalog. 
	%%\item {\em Filters:} Patterns can be filtered to return only a subset. 
	%%A pattern returns the set of primary key(s) for its rows matching some search criteria. A search criteria is specified via \emph{filters}. The sets of keys returned from two patterns can be combined using the union (or), intersection (and), set difference (not) and Symmetric difference (xor) operations.
	%%\item {\em Optimization:} \dsl~performs optimizations such as {\em rule-filtering} and {\em rule-rewriting} (Section~\ref{rule:rewriting}, \ref{rule:rewriting}). % and {\em magic sets}. %~\cite{Bancilhon:1985:MSO:6012.15399}. % used to avoid generating unnecessary facts. %The latter optimization is provided by the IRIS engine itself. 
%\end{enumerate}

%\textbf{Languages and Components.} 
The \dsl~framework comprises two primary languages: 
\begin{enumerate}
	%\item \emph{Operational Component:} This is the primary component for end-users to query data. They will use their domain knowledge and write code in the {\bf \dsl~language} syntax, which is the main language of the framework. 
	\item \emph{Query language:} This is the primary component, used for querying data with some user-specified criteria. This criteria is specified in what we call the \dsl~syntax.
	Its code is given in \texttt{typeface} font, as in: 
	\texttt{\bfseries def \#human as \#person}.
	
	\item \emph{Schema language:} 
	The initial schema is specified via {\bf \dcl}~(\dclfull). %More specifically, it is used to define the basis patterns. % for the domain. 
	%Its syntax is similar to Scala's function signatures. 
	\dcl~code is given in \dclcode{italics}, as in: \dclcode{person(name:String, age:Int)}. %Details are given in Section~\ref{configuration}.
	
	%\item 
  \end{enumerate}

%\textbf{Domain-Specific Configuration:} The \dcl~and Datalog code together form the {{\em domain-specific configuration}} or simply {\em configuration}.
%The configuration, whose details are given in Section~\ref{configuration}, will usuall be hidden from end-users and only edited by advanced users. % unless they are extending the framework. 

\textbf{Terminology:} We use the following terminology:
\begin{enumerate}
	\item \textbf{Tables:} \dsl~operates on primitives called {\em tables}. Table names always start with `\texttt{\#}' (example \texttt{\#person}). Users define new tables by extending existing ones. 
	%A pattern is analogous to a SQL table. 
	%Internally 
	%A able maps to a Datalog relation, or to a SQL table/view.
	\item \textbf{Basis:} These are initial tables defined via \dcl. 
	\item \textbf{Attributes:} A basis table has a name and one or more attributes. 
	%Attributes are equivalent to the columns of a database table.
	An attribute is a typed variable and starts with `\texttt{@}'. As an example, the \dcl~code: 
\begin{enumerate}
	\item[] \dclcode{person(name:String, age:Int, city:String)}
\end{enumerate}
defines a basis table \texttt{\#person} with three attributes: \texttt{@name}, \texttt{@city} (\texttt{String}s), and \texttt{@age} (\texttt{Int}). %\texttt{String} and \texttt{Int} are the only primitive types. 
User-defined tables do not have attributes. % but they can access attributes of basis tables via {\em traversal}.

\end{enumerate}
\full{
Figure~\ref{tab:\dsl} summarizes the various components. 
\begin{figure}[!h]
\centering
  \begin{tabular}{ |c|c|c|c| }\hline
 {\em Language~} & {\em Purpose}           & {\em User} & {\em Ref.}\\ \hline\hline
  			\dsl     & Define new tables     & Data seeker       & \ref{description}\\ \hline
	 		  \dcl     & Define initial tables & Data owner         & \ref{configuration}\\ \hline
		 	  %Datalog  & Define initial rules    & Advanced         & \ref{datalogrules}\\ \hline
  \end{tabular}
\caption{\dsl~framework components}
\label{tab:\dsl}
\end{figure}
}
%
%
%\paragraph{Reusability:} \dsl~encourages reusability. Multiple boot configurations can be specified as long as there are no violations of the semantic rules (Section~\ref{semantic}) in the combined one. As an example, the following \dsl~code demonstrates reuse of both configuration and other \dsl~code:\\
%\begin{small}
%\texttt{\phantom{1}~~~~basis my\_basis1.txt, my\_basis2.txt~~~~// load multiple basis configs}\\
%\texttt{\phantom{1}~~~~rules my\_rules1.txt, my\_rules2.txt~~~~// load multiple rules configs}\\
%\texttt{\phantom{1}~~~~import my\_code1.\dsl, my\_code2.\dsl~~~~~// reuse \dsl~code from here}
%\end{small}

\subsection{\dsl~Syntax}
\label{description}

\dsl~is the main language of the framework and allows users to do two things: (1) define new patterns by extending one or more existing patterns and (2) find rows in patterns. Here we explain both. 
%We use the configuration of Section~\ref{configuration} for the examples here.
\full{See Appendix~\ref{grammar} for the \dsl~grammar.}

\textbf{Defining Patterns:}
\label{def}
	A new pattern is defined as \\
\texttt{\phantom{1}~~~~~~~~~~~\small \underline{def <newPattern> as <match>}}\\
	Here \texttt{<match>} is a combination of one or more patterns and optional filters. We illustrate this via examples below. To improve readability in this section, \dsl~code will be in \colorbox{Light}{\small \texttt{gray}} and %In the following, 
	the \texttt{<match>} expression will be in \colorbox{Light}{\small \texttt{\bfseries{bold}}}. The patterns inside \texttt{<match>} are called the parents of \texttt{<newPattern>}. 
	
\begin{enumerate}
	\item \textbf{Filter:} A filter extracts a subset of a pattern. For instance: 	
\colorbox{Light}{\small \texttt{def \#adult as \bfseries{\#person where \{@age > 18\}}}}

A filter is a tuple ({\em attribute}, {\em operator}, {\em value}) (in this case (\texttt{@age}, \texttt{>}, \texttt{18})), specified using \texttt{where} keyword. % and enclosed in braces.\\
The \tld~ and \tld\tld~ operators represent wildcard and regex respectively, for use with \texttt{String}s.

	\item \textbf{Merging filters:} Multiple filters can be used, as in:\\
\texttt{\colorbox{Light}{\small def \#member as \bfseries{\#person where \{@age > 25 or }}} \\
\phantom{.}~~~~~~\texttt{\small \colorbox{Light}{\bfseries{(@city = 'Springfield' and @age > 20)\}}}}\\
	%where multiple filters are applied to \texttt{\#person}. 
	Filters are merged via \texttt{\bf and}/\texttt{\bf or} using parenthesis. 
	
	\item \textbf{Traversing patterns:} User-defined patterns do not have attributes. However, attributes of any basis pattern can be accessed using  \texttt{.} operator:
	
	\texttt{\small \colorbox{Light}{def \#elder as \bfseries{\#member where \{\#person.@age > 30\}}}}
	
	This is called traversal --  we `traverse' from \texttt{\#member} to \texttt{\#person} and access \texttt{@age}. %Traversing rules are given in Section~\ref{semantic}.
	
	%Note that some patterns are non-traversable (see Section~\ref{semantic}).\\
	
	\item \label{patterkeyexample}\label{pattern_as_value}\textbf{Patterns as values:} Some attributes can map to a pattern. Example, \texttt{@child} takes value \texttt{\#adult} in:
	
	\texttt{\small \colorbox{Light}{def \#senior as \bfseries{\#parent where \{@child = \#adult\}}}}
	
	The rules for this are described in section~\ref{patternkeys}.
	
	\item \label{end}\label{attribute_as_value}\textbf{Attributes as values:} Any attribute	can map to another attribute of the same type, as in: 
	
	\texttt{\colorbox{Light}{\small def \#MD as \bfseries{\#person where \{@name = \#corp.@founder\}}}}
%	where the attribute \texttt{@name} of \texttt{\#person} takes as value the attribute \texttt{@founder} of \texttt{\#corp}.

	\item \label{multi} \textbf{Merging patterns:} \texttt{<match>} can have multiple patterns merged via \texttt{\bf and},  \texttt{\bf or}, \texttt{\bf not} or \texttt{\bf xor} (denoting respectively, intersection, union, set difference and symmetric difference operations), as in: 
	\texttt{\colorbox{Light}{\small def \#eligible as }}\\
	\texttt{\colorbox{Light}{\small \bfseries{\{\#person where \{@age>20\} and \{\#member or \#MD\}\}}}}
	%\texttt{\phantom{.}~~~~~~~~~~~~~~~\colorbox{Light}{\bfseries{and \{\#member or \#director\}\}}}}.
	
	%where  the patterns \texttt{\#person} and \texttt{\#member} are joined using \textbf{and}. 
		Merged patterns must be enclosed in  braces. 
\end{enumerate}

%correctness guarantee. Always generates Safe datalog rules (that always terminate),


%Table~\ref{tab:operators} summarizes the various allowed operators in filters.
%\begin{table} [htbp]
	%\centering
		%\small
		%\texttt{\begin{tabular}{|c|c|c|c|}
		%\hline
%{\bf Operator}&{\bf Allowed Types}&{\bf Allowed Values}&{\bf Meaning}\\\hline\hline
%=, !=&Int, String, Pattern&value or attribute&usual meaning\\\hline
%\tld, \tld\tld&String&String&wildcards, regex\\\hline
%>, <, <=, >=&Int&value or attribute&usual meaning\\\hline
%\end{tabular}}
	%\caption{Allowed operators in filters}
	%\label{tab:operators}
%\end{table}	

\textbf{Finding Patterns:}
Results are returned using a \texttt{find} statement, the syntax of which is:
%\begin{enumerate}
%	\item[] 
\texttt{\textbf{find <match>}},
%\end{enumerate}
	where \texttt{<match>} is any single-pattern expression. %used in defining a pattern (see Section~\ref{def}, \ref{start}-\ref{end}). 
	For instance: \\
	\colorbox{Light}{\small \texttt{find \bfseries{\#eligible where \{\#person.@age < 50\}}}}.	
	
%Note that to improve readability, \texttt{find} does not directly accept multi-pattern \texttt{<match>} expressions (i.e., those joined via \texttt{and}, \texttt{or}, etc, as in Section~\ref{def}, \ref{multi}). To use such expressions, first define a new pattern using the expression and then use this new pattern in the \texttt{find} statement as we did above.

%\paragraph{Configuring \dsl~for a Domain:} \dsl~is not dependent on a domain. In order to use \dsl, it must be bootstrapped using an initial domain-specific configuration, which is done by an advanced user. Section~\ref{configuration} gives details of this configuration. Section~\ref{semantic} gives the semantic restrictions in \dsl~based on the configration. The configuration is validated to eliminate any bugs as described in Section~\ref{validation}.
%
%
\full{
The \dsl~compiler %(see Section~\ref{compilerlogic}) 
takes in a sequence of \texttt{def} and \texttt{find} statements along with schema definitions and code such as SQL or Datalog for some underlying database.
}
\subsection{Schema Definition}% via \dcl~and Datalog}
\label{configuration}
Recall that some initial patterns are mapped to the basis via configuration defined using \dcl~(\dclfull). Consider the earlier \dcl~example to define a basis pattern called \texttt{\#person}:\\ %(reproduced below) 
%Consider the example of Section~\ref{terminology}:\\\\
\phantom{.}~~~\dclcode{person(name:String, age:Int, city:String)~~~// \dcl~code}

In reality, the above will give an error because in addition to attributes, each pattern must have either one or more {\em primary keys} or one or more {\em pattern keys}. %These are essentially attributes with special meaning. 

\subsubsection{Primary Keys} 
\label{primarykeys}
%A primary key is equivalent to its counterpart in SQL - 

Two patterns are linked via primary keys. Any attribute can be set as a primary key by appending an ID enclosed in square brackets to the attribute, as in the following \dcl: \\ %. Therefore the following is valid \dcl~code (won't give an error):\\
{\dclcode{\phantom{~}~person(persID:String[pID], name:String, age:Int, city:String)}}

The above defines a primary key with ID \texttt{pID}, written as \texttt{[pID]}. This ID is used to match primary keys of other patterns. 
%(the identifier \texttt{persID} is never used).
Now let us define two more patterns via \dcl: \\
{ 
%\small
\dclcode{\phantom{~}~corp(corpID:String[cID], name:String, founder:String)}\\
\dclcode{\phantom{~}~works(persID:String[pID], firmID:String[cID], empID:String)}
}

Here \texttt{\#corp} has one primary key \texttt{[cID]}, while \texttt{\#works} has two primary 
 keys, \texttt{[cID]} and \texttt{[pID]}. The two patterns are connected via the common primary key \texttt{[cID]}, similar to the way RDBMS tables are connected. % via primary-foreign key relationships. 
However, there are no foreign keys in \dsl~and all such relationships are {\em many-to-many}. 
Primary keys have the following properties:
\begin{enumerate}
	\item Primary keys attributes cannot be used in a filter. They are used internally for traversal and merging. %ing patterns.
	%\item patterns connected via primary keys have a {\em many-to-many} relationship. 
	\item A pattern returns its primary keys (or pattern keys - see below). Therefore, \texttt{\#person} will return \texttt{[pID]}s. 
	\item A new pattern inherits the (primary or pattern) keys of its parent pattern(s). Therefore, any pattern defined using \texttt{\#person} will also return \texttt{[pID]}s. 
\end{enumerate}
%\textbf{Example.} 
%
%Now let us define two more patterns via \dcl: \\
%{ 
%%\small
%\dclcode{corp(corpID:String[cID], name:String, founder:String)}\\
%\dclcode{works(persID:String[pID], firmID:String[cID], empID:String)}\\
%}

\textbf{Traversal:} Traversal occurs when a pattern is extended and the filter accesses an attribute of a different pattern. The patterns must be connected by a chain of primary keys for traversal to be successful (otherwise an error will be thrown). If such a path exists then \dsl~will automatically join the relations (i.e., pattens) in the path using the primary keys.
As an example, the following pattern, \texttt{\#fromAcme} uses \texttt{\#corp} directly and \texttt{\#works} indirectly for traversal:\\ \colorbox{Light}{\small \texttt{def \#fromAcme as \bfseries{\#person where \{\#corp.@name='Acme'\}}}}.

 Traversal from \texttt{\#person} to \texttt{\#corp} is done via \texttt{\#works} as follows:
	{\small \texttt{\#person[pID] -> \#works[pID,cID] -> \#corp[cID]}}. 	
	The traversal is explicitly seen in the compiled Datalog code:\\	
	{\small \texttt{fromAcme(pID) :- person(pID, \_, \_, \_), works(pID, cID, \_),\\\phantom{1}~~~~~~~~~~~~~~~~corp(cID, 'Acme', \_).}}

	 %\texttt{\#fromAcme}'s keys are derived from its parent pattern (\texttt{\#person}). Thus, \texttt{\#fromAcme} has the primary key \texttt{[pID]}. %This attribute is hidden from the end-user because it is a primary key. 
	
	
	\subsubsection{Pattern keys} 
	\label{patternkeys}
	
	Pattern keys define attributes that map to another pattern (see Example~\ref{patterkeyexample}, Section~\ref{description}).
		%We cannot use a primary key for this, since the corresponding attribute is invisible. 
		Pattern keys are defined by % by appending 
	an ID enclosed in  braces. % to the attribute. 
	The following %\dcl~code 
	defines a pattern with two pattern keys both with ID \texttt{pID}:\\%, which we will write as \texttt{\{pID\}}:\\\\
	 \dclcode{\phantom{1}~~parent(person:String\{pID\}, child:String\{pID\})}
	
	A pattern key has the following features: %is like a primary key with some differences:
\begin{enumerate}
	\item A pattern key attribute can be used in a filter. It can map to any pattern that returns a primary key with the same ID. In our example, both \texttt{@child} and \texttt{@person} attributes of \texttt{\#parent} can map to any pattern that returns \texttt{[pID]} (such as \texttt{\#fromAcme}). Therefore, we can, for example, write: \\
	{\small \colorbox{Light}{\texttt{def \#AcmeParent as \#parent where \{@child = \#fromAcme\}}}}
	%	because \texttt{\#fromAcme} outputs \texttt{[pID]}.
	\item In a filter, a pattern returning a pattern key can be used in place of a pattern returning the same primary key. Therefore, in the code above the \texttt{@child} attribute can also take as input any pattern returning \texttt{\{pID\}}.
	\item Pattern keys are non-traversable and cannot be merged with primary keys. Thus, a pattern returning \texttt{[pID]} cannot be merged with one returning \texttt{\{pID\}}. %The next version of \dsl~will allow such joins provided the patterns return a single key.

\end{enumerate}

\textbf{Hiding returned pattern keys.} A pattern returns all its pattern keys. For instance, \texttt{\#parent} as defined above will return the pattern keys \texttt{\{pID, pID\}}. %Therefore, the \texttt{\#dependent} pattern defined above also returns same pattern keys because it uses \texttt{\#parent} in its \texttt{<match>}. 
Suppose, we want it to return only the first key (representing parents), we can skip the other key(s) by appending `\texttt{!}' to it, as in:\\
\dclcode{\phantom{1}~~~parent(person:String\{pID\}, child:String\{pID!\})}
%\phantom{1}~~~~~
%making \texttt{\#parent} return only the first \texttt{\{pID\}}. 

After this, \texttt{\#parent} will return only the first key \texttt{\{pID\}}, and  the following code, for example, will be valid \dsl:\\ %\\\\
 \colorbox{Light}{\small \texttt{def \#grandParent as \bfseries{\#parent where \{@child = \#parent\}}}}

%\subsubsection{Reduced basis}
%\label{reducedbasis} 
%Let us enumerate the basis patterns for the examples: \\
%{\small
%	\dclcode{\phantom{1}~~~person(persID:String[pID], name:String, age:Int, city:String)}\\
%	\dclcode{\phantom{1}~~~works(persID:String[pID], firmID:String[cID], empID:String)}\\
%	\dclcode{\phantom{1}~~~corp(corpID:String[cID], name:String, founder:String)}\\
%	\dclcode{\phantom{1}~~~parent(person:String\{pID\}, child:String\{pID\})}
%	}
%	
%The data for these patterns is supplied by the user. All such patterns form the  {\em reduced basis}.
%
%\subsubsection{Extended Basis}
%\label{extendedbasis} 
%Suppose we need to define patterns for siblings and ancestors. %, which cannot be done from \dsl. Therefore, 
%Let us add them to the basis via \dcl:\\
%{\small
%\dclcode{\phantom{1}~~~sibling(left:String\{pID\}, right:String\{pID\})}\\
%\dclcode{\phantom{1}~~~ancestor(ancestor:String\{pID\}, descendant:String\{pID\})}
%}
%
%Their data can be inferred from the reduced basis and should not be user-supplied. These form the {\em extended basis}.
%
%\subsubsection{Initial Rules}
%\label{datalogrules}
%
%Since the extended basis does not contain data, using it in \dsl~will return no results. 
%Therefore, some rules need to be supplied to populate it using the reduced basis. These are the {\em initial rules}. 
%
%In the case of Datalog, our initial rules would be:\\
%{\small
%\texttt{\phantom{1}~~~sibling(x,y) :- parent(p,x),~parent(p,y),~x != y.}\\
%\texttt{\phantom{1}~~~ancestor(x,y) :- parent(x,y).}\\
%\texttt{\phantom{1}~~~ancestor(x,y) :- ancestor(x,a), ancestor(a,y).}
%}
%
%In the case of SQL, our initial rules would map to views:\\
%{\small
%\texttt{\phantom{1}~~~sibling(x,y) :- parent(p,x),~parent(p,y),~x != y.}\\
%\texttt{\phantom{1}~~~ancestor(x,y) :- parent(x,y).}\\
%\texttt{\phantom{1}~~~ancestor(x,y) :- ancestor(x,a), ancestor(a,y).}
%}
%
%Once the reduced basis, the extended basis and the initial rules are defined, the configuration is complete. The configuration is validated as described in Appendix~\ref{validation}.
%
%A basis pattern can be hidden using `!' as in:\\
%\dclcode{works!(persID:String[pID],corpID:String[cID], empID:String)}
%
%Hidden basis patterns are used only for traversal.
%After this \texttt{\#works} will cannot be used in queries but may be used internally for traversal.

%\phantom{.}~~~\dclcode{person ! (name:String, age:Int, city:String)~~~// \dcl~code}


% and \dsl~queries can be issued. 
%Some additional configuration settings are given in Appendix~\ref{config}. 
Once the pattern schema, the primary and pattern keys are defined in \dcl, the schema definition is complete. The set of initial patterns defined in the schema is called the {\em basis}. Advanced users can make additional tweaks to the basis configuration as described in Appendix~\ref{basis}.

A \dsl~server will be initialized with a basis along with raw data for its patterns (supplied in CSV format). 
After this, it can accept \dsl~queries and return data based on the result mapping supplied. The \dsl~and the basis must adhere to semantic restrictions given in Section~\ref{semantic}. 

Since \dsl~is primarily a query language, it does not have queries to insert data and all external data must be provided in the initialize phase. However, every \texttt{def} query in \dsl~effictively creates a temporary pattern internally. 


%Section~\ref{compilerlogic} describes the compiler. 
%\section{Semantic Restrictions} 
\label{semantic}
		The following rules and restrictions must be followed with respect to the basis and \dsl~code.
\begin{enumerate}
	\item A basis table has at least one primary/secondary key and returns all its primary/secondary keys, with the exceptions described in Section~\ref{patternkeys}.
	%\item 
	\item A basis table cannot have both primary and secondary keys. Thus, a basis table with secondary key(s) is non-traversable.
	Using the definitions of Section~\ref{reducedbasis}, the following code will give an error:\\
	\texttt{\small \colorbox{Light}{def \#foo as \bfseries\#parent where \{\#person.@name='Homer'\}}}
	
This is because \texttt{\#parent} is non-traversable.
	%All other patterns must be traversible.   
	\item There cannot be multiple traversal paths in the basis. 
	The following \dcl~code will give an error:\\
{\small
\dclcode{person(persID:String[pID],~name:String,~age:Int,~city:String)}\\
\dclcode{works(persID:String[pID],corpID:String[cID],empID:String)}\\
\dclcode{corp(corpID:String[cID],~name:String,~founderID:String[pID])}
}

This is because there are two paths from \texttt{[pID]} to \texttt{[cID]}; one via \texttt{\#works} and another via \texttt{\#corp}. Such relations can be modeled using secondary keys. As an example, the following is valid \dcl~code:\\%a valid \dcl~configuration achieving essentially the same thing:\\
	{\small
\dclcode{person(persID:String[pID],~name:String,~age:Int,~city:String)}\\
\dclcode{works(persID:String\{pID\},corpID:String\{cID\},empID:String)}\\
\dclcode{corp(corpID:String[cID],~name:String,~founderID:String[pID])}
}
	\item Two tables can be merged (using \texttt{and}/\texttt{or}/\texttt{not}/\texttt{xor}) if only if their returned keys are identical. Using the basis defined in Section~\ref{reducedbasis}, the following will, therefore, give an error: \texttt{\small \colorbox{Light}{def \#foo as \bfseries{\{\#person and \#corp\}}}},
		
as \texttt{\#person} returns \texttt{[pID]}, while \texttt{\#corp} returns \texttt{[cID]}.
%Note that even %the %key types %(primary/secondary) 
%and their 
%ordering must be identical.
	\item A user-defined table returns the key(s) of its parent table(s). Therefore, a user-defined table is traversable if and only if its parents are traversable. As an example, \texttt{\#fromAcme} of Section~\ref{primarykeys} is traversable and we can write:\\ %\\\\
	\texttt{\small \colorbox{Light}{def \#emp1 as \bfseries\#fromAcme where \{\#works.@empID='1'\}}}
	
	\item Type restrictions are enforced in \dsl. Therefore, for example, using the definition of Section~\ref{primarykeys}, the following will given an error because an attribute of type \texttt{String} cannot be mapped to an \texttt{Int}.\\
	\texttt{\small \colorbox{Light}{def \#emp1 as \bfseries\#fromAcme where \{\#works.@empID = 1\}}}
\end{enumerate}

\textbf{Selecting The Basis.} %The basis must satisfy the above restrictions. 
First set the reduced basis to represent the initial data.
%, hiding some tables if needed (using `!'). 
Then set the extended basis for other relations that may be needed. If there are loops in traversal, make some tables non-traversable by using secondary keys. Finally, make those tables non-traversable whose attributes need to map to other tables. 

A basis table can be hidden from the user using `!' as in:\\
\dclcode{works!(persID:String[pID],corpID:String[cID], empID:String)}.

Such hidden tables cannot be used in DQL. However, they might be used internally for traversal.

%Hidden tables is useful for intermediate definitions. 

%\paragraph{\dsl~Compiler:} 
%\paragraph{Configuration Validation.}\label{para:rulevalidation}
%The configuration is validated to eliminate any errors in the basis or the initial rule definitions. Appendix~\ref{validation} describes the various checks.
%\paragraph{Compilation and Optimization.}\label{compiler}
%The compiler converts \dsl~code into Datalog rules that can be fed to a Datalog engine. 
%The compiler is written entirely in Scala 
%~\cite{Odersky:2011:PSC:1983576} 
%Initially \dsl~code is parsed into an AST 
%using ANTLR~\cite{antlr}. %, which is then used to generate the Datalog code. 
%Section~\ref{compilerlogic} describes the compilation logic of \dsl. %In addition, %\dsl~additionally provides the following optimizations: 
%\begin{itemize}
%	\item 
%{\em Rule filtering} %Rule filtering 
	%is used to avoid evaluating unnecessary rules. From the initial Datalog rules, only the rules reachable from the queries are evaluated.  
%	\item 
%{\em 
%Magic Sets:} 
%Magic sets}~\cite{Bancilhon:1985:MSO:6012.15399} are used to avoid generating unnecessary facts. %The latter optimization is provided by the IRIS engine itself. 
%\end{itemize}

\input{dsl/analysisWshop.tex}
\subsection{Result Mapping}

%\textbf{Results Mapping:} 
A DQL \texttt{find} query returns the primary or pattern keys of the pattern. These can be considered as unique row IDs. The query \texttt{\small \underline{find \#users where \{\#state.@capital = `Boston`\}}} could, for example, return following keys of type \texttt{[userID]}:\\
{\small
	\texttt{\phantom{~}user01}% \\
	%\texttt{\phantom{~}user1367} \\
%	\texttt{\phantom{~}user04} 
}

These are not meaningful to end-users. The above row IDs can be mapped to meaningful values using a {\em Result Mapping}. %, also defined by the data owner. 
%For instance, in a query returning userIDs, the result mapping might instead cause it to return user names. 
As an example, the following lines define various result mappings from \texttt{userID}:\\
{\small
	\texttt{\phantom{~}map :a as \$userID => \#users // all public columns} \\
	\texttt{\phantom{~}map :b as \$userID => \#users.@name // name}\\
	\texttt{\phantom{~}map :c as \$userID => \#users.@age.avg // avg of age}
%	\texttt{\phantom{~}map :d as \$userID => \#users.@age.max // max of age}\\
%	\texttt{\phantom{~}map :e as \$userID => \#users.count // all row count}
%	\texttt{\phantom{~}map :f as \$userID => \#users.@age.count // distinct(age) count}
}

We can then specify a mapping in the = query as follows:\\ 
\texttt{\small \underline{find \#users:a where \{\#state.@capital = `Boston`\}}}. The query would then return the following:\\
{\small
	\texttt{\phantom{~}name |age}\\
	\texttt{\phantom{~}----------|------}\\
	\texttt{\phantom{~}Homer|45 }%\\
%	\texttt{\phantom{~}Bart |11}  
}

The grammar for mapping is given in Appendix~\ref{mapping}.
%
%Similarly, \texttt{\small \underline{find \#users:c where \{\#state.@capital = `Springfield`\}}} would output:\\
%{\small
%\texttt{\phantom{~}28}  
%}

\section{Restricting Queries}

\textbf{Constraint Language:} In OPAL, we also need a way to restrict the type of queries to prevent data leakage. This is done by constraining the kind of DQL queries allowed. We can, for example, restrict certain columns of certain  patterns (e.g., {\em age} in pattern {\em users}) and the operations on those columns (cannot do $<, >, \geq, \leq, \neq$). Additionally, the result mapping can be used to restrict the final view of the query results.

When specifying constraints, instead of following the standard blacklist approach ({\em deny these and allow rest}), we follow a whitelist one ({\em allow these and deny rest}). 
Thus, the constraints define the allowed (rather than denied) queries, and are usually created by the data-owner. If no constraints are specified, then all queries are disallowed. As an example, 
the following allows DQL queries that access the column {\em age} of pattern {users} with the operations $=, \neq$:\\
\texttt{\phantom{~}~ageConstraint: \#users.@age: =, !=}


%That is, we specify what type of queries are allowed a

When making a query, the following needs to be specified:
\begin{enumerate}
	\item A set $Q$ of DQL queries of \texttt{def} and \texttt{find} statements.
	\item A set $C$ of whitelist constraints that accept the queries.
	\item An set $R$ of result mappings, one for each \texttt{find} query. 	
\end{enumerate}



%DQL also allows constraints to be placed which queries a pattern can be used in (set by the owner of the pattern), and how a pattern can be combined with other patterns).

%\begin{tabular}{ l c r }
%DQL & SQL & Datalog \\\hline
%Pattern & Table or view & Relation \\
%Query & SQL & Rule \\
%\end{tabular}

%Comparison between concepts in DQL, SQL and Datalog.

\section{Conclusion}
\label{conclusion}

%Datalog is a language for querying deductive databases. However, it has has an obscure syntax. 
%This makes it difficult to maintain and verify Datalog rules. 
We described \dslfull~(\dsl), a DSL that automates the generation and maintenance of generic database queries in a human-friendly syntax. [...]
 
%Unlike similar tools, \dsl~is domain-independent and does not have pre-defined relations. Instead, a domain-expert defines such relations via the {\em basis} and {\em initial rules}.  
%The \dsl~compiler 
%outputs only safe rules~\cite{revesz1990closed} and 
%\dsl~provides correctness guarantees due to type-checking and validation. % of initial rules against the basis.

%\dsl~can be used in any domain where Datalog is used. One example is static analysis~\cite{Whaley:2005:UDB:2099708.2099719,dataloganalysis11,
%Huang:2011:DEA:1989323.1989456,
%Lam:2005:CPA:1065167.1065169,
%saxena13soql}.
%,pql,Crew:1997:ALE:1267950.1267968,
%cq,
%jqry}. 
%To this end, we presented a static analysis framework for Java using \dsl, which we call J\dsl. We showed how various bug-patterns (such as SQL injection) can be succinctly expressed in J\dsl. Since the framework works on bytecode (in addition to source), it can also be used for other JVM-based languages. 

%Some potential enhancements to \dsl~are:
%%Some possible enhancements of \dsl~are given below:
%\begin{enumerate}
	%\item Use rule-rewriting. 
	%%~\cite{DBLP:journals/toplas/LiuS09}. 
	%For example, in joins, several rules can be combined to avoid extra computation.
	%\item Use Binary Decision Diagrams for efficiency as in~\cite{Whaley:2005:UDB:2099708.2099719}. 
	%%\item Use Datalog with aggregates, such as a `count'. % (e.g., \cite{datomic}). 
%\end{enumerate}

%\bibliographystyle{abbrvnat}
\bibliographystyle{unsrt}
\bibliography{dsl/DSLbib}

\appendix

\section{Overview of Datalog}
\label{datalog-intro}
\label{Datalog} 
Datalog is a declarative programming language 
used for querying deductive databases.
%~\cite{Ramakrishnan93asurvey} 
It has found applications in information extraction, program analysis and security in recent years~\cite{Huang:2011:DEA:1989323.1989456,datalog,Ramakrishnan93asurvey}.
%cite{datalog}
%Unlike SQL, Datalog queries express the WHAT rather than the HOW. 
In Datalog, table schemas -- for example, \texttt{\small Parent(A, B)} -- are called as {\em Relations} and the corresponding table rows -- for example, \texttt{\small Parent('Homer', 'Bart')} -- are called {\em Facts}. 
New tables are constructed via {\em Rules}. Rules are made of two parts separated by the \texttt{:-} symbol: (1) One {\em Head relation} on the left, and (2) One or more {\em Tail relations} on the right each separated by the \texttt{,} symbol. For example:\\
		{\small\phantom{.}~~\texttt{GrandParent(A, C) :- Parent(A, B), Parent(B, C)} }
	% \item Two head relations are identical if they have the same name and number of columns.
	
	Datalog supports recursive rules such as: \\
		{\small\phantom{.}~~\texttt{Ancestor(A, C) :- Ancestor(A, B), Ancestor(B, C)}}
	
	Tail relations in the same rule are combined using conjunction. Two or more rules with the same head relation -- for example below -- are combined using disjunction:\\
{\small\phantom{.}~~\texttt{Ancestor(A, B) :- Parent(A, B)}\\
\phantom{.}~~\texttt{Ancestor(A, C) :- Ancestor(A, B), Ancestor(B, C)}}
	
\textbf{Comparison with Relational Algebra.} Although Datalog has similarities to SQL/relational algebra, they are different. 
Any expression in basic relational algebra can be expressed in Datalog. Operations in extended relational algebra (grouping, aggregation, sorting) are not supported in Datalog. Datalog can express recursion, which SQL cannot.

\section{\dsl~Grammar in ANTLR}
\label{grammar}

The \dsl~compiler is written in Scala using ANTLR~v3~\cite{antlr}, a tool for generating parsers. The following is the \dsl~grammar, created using ANTLRWorks (\url{www.antlr3.org/works}). 
%The following is the grammar for \dsl~in ANTLR~\cite{antlr}, a popular tool for generating parsers. 
\begin{small}
\begin{verbatim}
//  Lexer rules start with uppercase letter
//  Parser rules start with lowercase letter.
//  
// .. creates a character range
// + means "repeat one or more times". 
// * means "repeat zero or more times".    
// ? means "optional" (i.e., "repeat zero or one time")
// | means "or"    

tables:       'tables' ID ':' (define|find)* EOF;
find:         'find' simpleMatch;
define:       'def' TABLE 'as' tableMatch;
tableMatch:   simpleMatch | ('{' complexMatch '}');
simpleMatch:  TABLE filter?;
complexMatch: tableMatch boolMatch tableMatch;
filter:       'where' '{' rule '}';
rule:         simpleRule | complexRule;
mixedRule:    simpleRule | ('(' complexRule ')');
simpleRule:   attValStr | attValInt | attValTab | attValAtt;
complexRule:  mixedRule bool mixedRule;
boolMatch:    bool | 'not' | 'xor';
bool:         'and' | 'or';
attValStr:    ATT opStr STRING;
attValInt:    ATT opInt INT;
attValAtt:    ATT opEq ATT;
attValTab:    ATT opEq TABLE;
opEq:         '=' | '!=';
opInt:        opEq | '<=' | '>=' | '<' | '>' ;
opStr:        opEq | '~' | '~~';
ATT:          '@' ID | (TABLE '.' '@' ID);
TABLE:        '#' ID;
//ID,STRING,INT are defaults defined with AntlrWorks
//ID: anything starting with non-number, INT: number
//STRING: Anything enclosed in single quotes
\end{verbatim}
%(www.antlr3.org/works)
\end{small}
\section{Datalog Configuration}
\label{basis}
A \dsl~database uses Datalog as its underlying query engine. Thus, advanced users familiar with Datalog can make additional use of its powerful features (such as recursion). Recall that the basis is the set of initial table schema. Here we describe how the basis can be enhanced with Datalog. First we split the basis into a {\em reduced basis} and an {\em extended basis}. The reduced basis is essentially the schema of tables mapping to the externally supplied raw data. The extended basis is derived from the reduced basis via Datalog rules and does not correspond to externally supplied data. We explain this by extending the earlier examples.

\subsection{Reduced basis}
\label{reducedbasis} 
Let us enumerate the basis tables for the examples of Sections~\ref{description} and \ref{configuration}: \\
{\small
	\dclcode{\phantom{1}~~~person(persID:String[pID], name:String, age:Int, city:String)}\\
	\dclcode{\phantom{1}~~~works(persID:String[pID], firmID:String[cID], empID:String)}\\
	\dclcode{\phantom{1}~~~corp(corpID:String[cID], name:String, founder:String)}\\
	\dclcode{\phantom{1}~~~parent(person:String\{pID\}, child:String\{pID\})}
}

The data for these tables is supplied by the user. All such tables form the  {\em reduced basis}.

\subsection{Extended Basis}
\label{extendedbasis} 
Suppose we need to define tables for siblings and ancestors. %, which cannot be done from \dsl. Therefore, 
Let us add them to the basis via \dcl:\\
{\small
	\dclcode{\phantom{1}~~~sibling(left:String\{pID\}, right:String\{pID\})}\\
	\dclcode{\phantom{1}~~~ancestor(ancestor:String\{pID\}, descendant:String\{pID\})}
}

Their data can be inferred from the reduced basis and should not be user-supplied. These form the {\em extended basis}.

\subsection{Initial Rules}
\label{datalogrules}

Since the extended basis does not contain data, using it in \dsl~will return no results. 
Therefore, some rules need to be supplied to populate it using the reduced basis. These are the {\em initial rules}. 

In the case of Datalog, our initial rules would be:\\
{\small
	\texttt{\phantom{1}~~~sibling(x,y) :- parent(p,x),~parent(p,y),~x != y.}\\
	\texttt{\phantom{1}~~~ancestor(x,y) :- parent(x,y).}\\
	\texttt{\phantom{1}~~~ancestor(x,y) :- ancestor(x,a), ancestor(a,y).}
}

In the case of SQL, our initial rules would map to views:\\
{\small
	\texttt{\phantom{1}~~~sibling(x,y) :- parent(p,x),~parent(p,y),~x != y.}\\
	\texttt{\phantom{1}~~~ancestor(x,y) :- parent(x,y).}\\
	\texttt{\phantom{1}~~~ancestor(x,y) :- ancestor(x,a), ancestor(a,y).}
}

Once the reduced basis, the extended basis and the initial rules are defined, the configuration is complete. The configuration is validated as described in Appendix~\ref{semantic}.


%\input{dsl/additionalconfigWshop.tex}
%\section{Semantic Restrictions} 
\label{semantic}
		The following rules and restrictions must be followed with respect to the basis and \dsl~code.
\begin{enumerate}
	\item A basis table has at least one primary/secondary key and returns all its primary/secondary keys, with the exceptions described in Section~\ref{patternkeys}.
	%\item 
	\item A basis table cannot have both primary and secondary keys. Thus, a basis table with secondary key(s) is non-traversable.
	Using the definitions of Section~\ref{reducedbasis}, the following code will give an error:\\
	\texttt{\small \colorbox{Light}{def \#foo as \bfseries\#parent where \{\#person.@name='Homer'\}}}
	
This is because \texttt{\#parent} is non-traversable.
	%All other patterns must be traversible.   
	\item There cannot be multiple traversal paths in the basis. 
	The following \dcl~code will give an error:\\
{\small
\dclcode{person(persID:String[pID],~name:String,~age:Int,~city:String)}\\
\dclcode{works(persID:String[pID],corpID:String[cID],empID:String)}\\
\dclcode{corp(corpID:String[cID],~name:String,~founderID:String[pID])}
}

This is because there are two paths from \texttt{[pID]} to \texttt{[cID]}; one via \texttt{\#works} and another via \texttt{\#corp}. Such relations can be modeled using secondary keys. As an example, the following is valid \dcl~code:\\%a valid \dcl~configuration achieving essentially the same thing:\\
	{\small
\dclcode{person(persID:String[pID],~name:String,~age:Int,~city:String)}\\
\dclcode{works(persID:String\{pID\},corpID:String\{cID\},empID:String)}\\
\dclcode{corp(corpID:String[cID],~name:String,~founderID:String[pID])}
}
	\item Two tables can be merged (using \texttt{and}/\texttt{or}/\texttt{not}/\texttt{xor}) if only if their returned keys are identical. Using the basis defined in Section~\ref{reducedbasis}, the following will, therefore, give an error: \texttt{\small \colorbox{Light}{def \#foo as \bfseries{\{\#person and \#corp\}}}},
		
as \texttt{\#person} returns \texttt{[pID]}, while \texttt{\#corp} returns \texttt{[cID]}.
%Note that even %the %key types %(primary/secondary) 
%and their 
%ordering must be identical.
	\item A user-defined table returns the key(s) of its parent table(s). Therefore, a user-defined table is traversable if and only if its parents are traversable. As an example, \texttt{\#fromAcme} of Section~\ref{primarykeys} is traversable and we can write:\\ %\\\\
	\texttt{\small \colorbox{Light}{def \#emp1 as \bfseries\#fromAcme where \{\#works.@empID='1'\}}}
	
	\item Type restrictions are enforced in \dsl. Therefore, for example, using the definition of Section~\ref{primarykeys}, the following will given an error because an attribute of type \texttt{String} cannot be mapped to an \texttt{Int}.\\
	\texttt{\small \colorbox{Light}{def \#emp1 as \bfseries\#fromAcme where \{\#works.@empID = 1\}}}
\end{enumerate}

\textbf{Selecting The Basis.} %The basis must satisfy the above restrictions. 
First set the reduced basis to represent the initial data.
%, hiding some tables if needed (using `!'). 
Then set the extended basis for other relations that may be needed. If there are loops in traversal, make some tables non-traversable by using secondary keys. Finally, make those tables non-traversable whose attributes need to map to other tables. 

A basis table can be hidden from the user using `!' as in:\\
\dclcode{works!(persID:String[pID],corpID:String[cID], empID:String)}.

Such hidden tables cannot be used in DQL. However, they might be used internally for traversal.

%Hidden tables is useful for intermediate definitions. 

%
%\section{Configuration Validation}
%\label{validation}
%%\paragraph{Initial Rule Validation.} 
%The domain-specific configuration comprises three things: (1) Reduced basis, (2) Extended basis, and (3) Datalog rules to populate (2) from (1). 
%%the extended basis from the reduced basis. 
%%Errors can creep in at any of these. 
%Each is prone to errors. 
%For instance, subtle bugs can arise to typos in hand-written Datalog rules. 
%%Some of these are difficult to debug because the Datalog engine does not catch them. 
%%We found that a majority of the time is spent in debugging such errors. 
%\dsl~validates the Datalog rules against the basis to eliminate several errors described below: 
%%not caught by the Datalog engine.
%%\footnote{One error that Datalog does catch is {\em unlimited variables} (resulting in {\em unsafe rules}~\cite{revesz1990closed}). As an example, the rule \texttt{Ancestor(x, y) :- Person(x).} is unsafe because of the unlimited variable \texttt{y} (it does not appear on the right side).} 
%%The errors caught are given below:
%\begin{enumerate}
	%\item \emph{Reduced basis with rules:} There should not exist rules with head relations from the reduced basis. Reduced basis should be populated only via user input. 
%
	%\item \emph{Extended basis without rules:} Every pattern in the extended basis must have at least one rule where it is the head relation. Otherwise, such patterns will not be populated. This can occur due to typos in the Datalog rules. As an example, refer to the initial rules of Section~\ref{datalogrules}.  If the first rule is mistyped as\\
	%{\small \texttt{\colorbox{Faint}{siblng}(x,y) :- parent(p,x), parent(p,y), x!=y.}}\\
%Then the \texttt{\#sibling} pattern will not be populated.	
%
	%\item \emph{Undefined tail relation:} Every tail relation must either refer to a pattern in reduced basis or to another head relation. Otherwise, any rule that uses the relation will return no data. As an example, consider the earlier rule, but now mistyped as\\
	%{\small \texttt{sibling(x,y) :- parent(p,x),\colorbox{Faint}{parnt}(p,y), x!=y.}}
	%
%Then the \texttt{\#sibling} pattern will not be populated.	
	%
  %\item \emph{Cyclic definitions:} For every relation, there must be at least one rule where it is a head relation and not the tail relation. Otherwise such relations will not contain data. As an example, refer to Section~\ref{datalogrules}. Suppose the second rule is mistyped as\\
	%\texttt{\small \colorbox{Faint}{ancestr}(x,y) :- parent(x,y).}\\
	%Then only the third rule, given below, is used:\\ % for \texttt{\#ancestor}.\\\\
%{\small \texttt{ancestor(x,y) :- ancestor(x,a), ancestor(a,y).}}\\
%This will have no data because the relation is cyclic. 
%
 %%(i.e., defined using only itself). This can also occur %(and will be caught) 
%%if the mistyped rule is absent altogether. %The compiler checks for cycles only within a single rule, and not across multiple rules. 
  %\item \emph{Unused head relations:} Every head relation must map to either a pattern in the extended basis or to a tail relation of another rule. A typo (such as the example in Section~\ref{limitations}, where the \texttt{Parent} relation is never used) can cause this to be violated. 
	%%This will also catch cyclic error in the previous example due to the mistyped rule but not due to the absence of the rule. 
  %\item \emph{Mismatch in basis:} The number and types of parameters in the rules must match the basis. Consider the first rule of Section~\ref{datalogrules} mistyped as:\\
		%{\small 
		%\texttt{sibling(x,y) :- parent(p,x), parent(p,y), x\colorbox{Faint}{>}y.}\\
		%}
%(use of `\texttt{>}' instead of `\texttt{!=}'). This causes \texttt{x, y} to be wrongly inferred as integers, causing incorrect data. 
%
%%Not only the types but also the number of parameters are checked, so the following error will also be caught:\\
%%{\small \texttt{sibling\colorbox{Faint}{(x)} :- parent(p,x), parent(p,y), x != y.}}
%
%\item \emph{Mismatch in rules:} 
%%The above finds mismatched parameters for relations in the basis. However, 
%There may be intermediate relations not in the basis. % of other relations in the basis. 
%For example, \texttt{secondCousin} in the extended basis 
%could be defined using an intermediate relation \texttt{grandParents} not in the basis. This check finds mismatched parameters in intermediate relations.
%
%\item \emph{Incorrect wildcard usage:} \label{wildcard}\dsl~considers any variables starting with the text `\texttt{ANY}' as wildcards and will throw an error if such variables are repeated in a rule. 
%\end{enumerate}
%
%% by ensuring that a head rule indeed exists either in the reduced basis or in the initial rules corresponding to every tail rule. This checking is performed over parameter types and number. It additionally validates that for every pattern in the extended basis, there is at least one rule with that pattern as the head relation. %If all checks pass, it proceeds to compilation. 
%
%
%%\paragraph

%\input{dsl/compilationLogicWshop.tex}
%\input{dsl/javabasis.tex}
%\input{dsl/quickstart}
%\acks
\section{Semantic Restrictions} 
\label{semantic}
		The following rules and restrictions must be followed with respect to the basis and \dsl~code.
\begin{enumerate}
	\item A basis table has at least one primary/secondary key and returns all its primary/secondary keys, with the exceptions described in Section~\ref{patternkeys}.
	%\item 
	\item A basis table cannot have both primary and secondary keys. Thus, a basis table with secondary key(s) is non-traversable.
	Using the definitions of Section~\ref{reducedbasis}, the following code will give an error:\\
	\texttt{\small \colorbox{Light}{def \#foo as \bfseries\#parent where \{\#person.@name='Homer'\}}}
	
This is because \texttt{\#parent} is non-traversable.
	%All other patterns must be traversible.   
	\item There cannot be multiple traversal paths in the basis. 
	The following \dcl~code will give an error:\\
{\small
\dclcode{person(persID:String[pID],~name:String,~age:Int,~city:String)}\\
\dclcode{works(persID:String[pID],corpID:String[cID],empID:String)}\\
\dclcode{corp(corpID:String[cID],~name:String,~founderID:String[pID])}
}

This is because there are two paths from \texttt{[pID]} to \texttt{[cID]}; one via \texttt{\#works} and another via \texttt{\#corp}. Such relations can be modeled using secondary keys. As an example, the following is valid \dcl~code:\\%a valid \dcl~configuration achieving essentially the same thing:\\
	{\small
\dclcode{person(persID:String[pID],~name:String,~age:Int,~city:String)}\\
\dclcode{works(persID:String\{pID\},corpID:String\{cID\},empID:String)}\\
\dclcode{corp(corpID:String[cID],~name:String,~founderID:String[pID])}
}
	\item Two tables can be merged (using \texttt{and}/\texttt{or}/\texttt{not}/\texttt{xor}) if only if their returned keys are identical. Using the basis defined in Section~\ref{reducedbasis}, the following will, therefore, give an error: \texttt{\small \colorbox{Light}{def \#foo as \bfseries{\{\#person and \#corp\}}}},
		
as \texttt{\#person} returns \texttt{[pID]}, while \texttt{\#corp} returns \texttt{[cID]}.
%Note that even %the %key types %(primary/secondary) 
%and their 
%ordering must be identical.
	\item A user-defined table returns the key(s) of its parent table(s). Therefore, a user-defined table is traversable if and only if its parents are traversable. As an example, \texttt{\#fromAcme} of Section~\ref{primarykeys} is traversable and we can write:\\ %\\\\
	\texttt{\small \colorbox{Light}{def \#emp1 as \bfseries\#fromAcme where \{\#works.@empID='1'\}}}
	
	\item Type restrictions are enforced in \dsl. Therefore, for example, using the definition of Section~\ref{primarykeys}, the following will given an error because an attribute of type \texttt{String} cannot be mapped to an \texttt{Int}.\\
	\texttt{\small \colorbox{Light}{def \#emp1 as \bfseries\#fromAcme where \{\#works.@empID = 1\}}}
\end{enumerate}

\textbf{Selecting The Basis.} %The basis must satisfy the above restrictions. 
First set the reduced basis to represent the initial data.
%, hiding some tables if needed (using `!'). 
Then set the extended basis for other relations that may be needed. If there are loops in traversal, make some tables non-traversable by using secondary keys. Finally, make those tables non-traversable whose attributes need to map to other tables. 

A basis table can be hidden from the user using `!' as in:\\
\dclcode{works!(persID:String[pID],corpID:String[cID], empID:String)}.

Such hidden tables cannot be used in DQL. However, they might be used internally for traversal.

%Hidden tables is useful for intermediate definitions. 

%We would like to acknowledge Aritra Dhar for his work on the Soot framework.
\section{Mapping Grammar in ANTLR}
\label{grammar}

The following is the mapping grammar in ANTLR3. 
\begin{small}
	\begin{verbatim}
	mappings:    'mappings' ID  (mapping)+ EOF;
	mapping:     'map' MAP 'as' keyMaps ';';
	MAP:         ':' ID;
	keyMaps:     keyMap | (keyMap ',' keyMaps);
	keyMap:      KEY^ '=>' value;
	value:       COUNT | pattern;
	pattern:     PATTERN ('.' countOrAttr (',' countOrAttr)*)?;
	countOrAttr: COUNT | attribute;
	attribute:   ATTR^ ('.' countOrAggr)?;
	countOrAggr: COUNT | AGGR;
	PATTERN:     '#' ID;
	KEY:         '$' ID;
	ATTR:        '@' ID;
	AGGR:        'max'|'min'|'sum'|'avg';	
	COUNT:       'count';	
	\end{verbatim}
\end{small}

\end{document}
